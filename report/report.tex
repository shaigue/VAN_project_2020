\documentclass{article}
\usepackage{amsmath}

\title{VAN final project report}
\date{2020-05-01}
\author{Shai Guendelman, Avnom Hanohov}

\begin{document}
	% this is how to comment with '%'
	% this produces the title in a single front page
	\maketitle
	\newpage
	
	% sections, subsections, and subsubsections are of 
	% different topics.
	% paragraphes and subparagraphes are not numbered
	% this is a section with latex examples. delete this before 
	% final draft, but keep to use as referance
	\section{Latex examples section - delete after}
	\subsection{this is a subsection}
	sum subsection stuff 
	
	\paragraph{this is unnumbered paragrapha, can go anywere}
	some staff
	\subsubsection{you can even go 2 levels down}
	\paragraph{even here!}
	paragraph. you can force line-down with \\
	down!
	\\
	you can put anywere in the text some math by using the $$double dollar$$ or $dollar$ sign use inline math \\
	you can also do numbered equations:
	\begin{equation}
	x=y
	\end{equation}
	or un-numbered equations:
	\begin{equation*}
	x=y
	\end{equation*}
	if you want to stack equations on top of each other, and align them, you can use the "align" and "align*" for unnumbered, and each line will be aligned so that the '\&' symbol will be under the first eq.
	\begin{align}
	shai&=present \\
	potato&=tapud	
	\end{align}
	examples for subscript, superscript, fraction, sqrt and integral:
	\begin{align*}
	a &= x^2 \\
	b &= x_1^2 + x_{i_1}^{j_2} \\
	c &= a - x_j \\ 
	d &= \int p(x)xdx \\
	f &= \frac{num}{den} \\
	s &= \sqrt{2}
	\end{align*}
	we can add brakets that hold a certain object and size:
	\begin{equation*}
	\left[
	\begin{matrix}
	11 & 12 & 13 \\
	21 & 22 & 23 \\
	31 & 32 & 33
	\end{matrix}
	\right]
	\end{equation*}
	or :
	\begin{equation*}
	\left(
	\begin{matrix}
	11 & 12 & 13 \\
	21 & 22 & 23 \\
	31 & 32 & 33
	\end{matrix}
	\right)
	\end{equation*}
	Greek letters usually with writing their name: $\lambda$ with small first letter for lower case, and for upper case use $\Lambda$ with upper case first letter
	\\
	this is enough for now, want to learn more go to https://www.latex-tutorial.com/tutorials/figures/ 
	to continue\\
	
	\section{Introduction and overview}
	% TODO: Introduce the paper(s) topic. Describe how the paper fits in with the contents of this course, provide a brief background (literature review) and explain why the problem is important
	\section{Preliminary material and problem formulation}
	% TODO: Present a description of relevant notations and definitions, define mathematically the problem addressed by the paper(s), and summarize any preliminary mathematical material used in the paper(s).
	\section{Main contribution}
	% TODO: A detailed discussion of the main results of the paper(s). This should include both a qualitative discussion and a mathematical presentation (i.e. show proofs, preferably in your own style).
	\section{Implementation}
	% TODO: Demonstrate the main results of the paper(s) using simulation and/or real-world experiments. You are free to choose the programming language as well as using open source software. This also includes testing the approaches under different conditions than those originally assumed in the paper(s), as well as extending approaches to unsupported settings/scenarios.
	\section{Discussion and Conclusions}
	% TODO: Summarize the report and provide some criticism: identify weak points, unrealistic assumptions or aspects that could be improved and suggest possible directions (or extensions) for future research
	
  
\end{document}
